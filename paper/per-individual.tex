% This was reduced substantially in the paper; here is the original content.
\subsection*{Per-individual analysis.}

Because the linear mixed effects model showed that both the condition
and the individual contributed a similar amount of variation to the
latency, we conducted an analysis of latency values within each
individual to find any behavioural clustering. We were interested in
finding out whether the two different distractor conditions might
require distinct aptitudes from the subjects. We tried to cluster
individuals based on an analysis of whether the individual was
statistically significantly faster to react in the `Asynchronous
Distractor' condition compared with the `Synchronous Distractor'
condition.

\begin{table}[ht]
\caption{Mean error rates in the three conditions for the entire group
  of 55 subjects, and the three identified clusters (of sizes 13, 36
  and 6 subjects) showed some evidence of a speed-accuracy tradeoff. Mean
  error rates are given with the 95\% confidence interval (computed
  with 100 bootstrap replications) shown in brackets (the interval is
  +/- this value).}  \centering
\begin{tabular}{c c c c}
\hline
\raisebox{-3ex}{\textbf{Group}} & \multicolumn{3}{c}{\raisebox{-1ex}{\textbf{Error rates for conditions}}} \\ [1ex]
 & Synchronous & No Distractor & Asynchronous \\ [1ex]
\hline
\textbf{\emph{Entire group (N=55)}} & \emph{ 0.125 (0.016)} & \emph{0.016 (0.005)} &  \emph{0.302 (0.029)} \\
\textbf{Faster Synchronous (N=13)}  &        0.143 (0.029)  &       0.016 (0.009)  &        0.267 (0.046)  \\
\textbf{No Difference (N=36)}       &        0.125 (0.025)  &       0.016 (0.007)  &        0.314 (0.045)  \\
\textbf{Faster Asynchronous (N=6)} &        0.089 (0.052)  &       0.018 (0.011)  &        0.301 (0.078)  \\ [1ex]
\hline
\end{tabular}
\label{table:clustering}
\end{table}

Based on a comparison using bootstrap analyses of each individual's
movement latencies, we found a group of 13 individuals who were
significantly faster to react in the synchronous condition (`Faster
Synchronous') and a group of 6 individuals who were faster in the
asynchronous condition (`Faster Asynchronous'). An individual was
deemed to be a member of the faster synchronous group if the upper
bound of the bootstrap-determined 90\% confidence interval about the
mean of the synchronous distractor latencies was smaller than the
lower bound of the 90\% confidence interval from the asynchronous
distractor latencies. The probability of a Type I error (that is of
having mis-classified the individual) is $0.1^2 = 0.01$. The remaining
36 individuals showed no significant difference in reaction time
between these two distractor conditions.

Table~\ref{table:clustering} shows the error rates for the different
groups. Group membership had no significant effect on the error
rates. Table~\ref{table:clustering_latencies} shows mean latencies
within the different groups. The `Faster Synchronous' group had a
faster movement response compared with the `No Difference' group in
the synchronous distractor condition; no significant difference in the
absence of a distractor and a slower movement response in the
asynchronous distractor condition.

Again, compared with the `No Difference' group, the `Faster
Asynchronous' group was significantly slower to move in the synchronous
distractor condition although in the asynchronous distractor
condition, these subjects were not significantly faster or slower than
the `No Difference' group. This group was slower to respond in the
no distractor condition.

These results suggest that some individuals appear to be better tuned
to ignoring asynchronous distractors, whereas others are able to make
faster, accurate movement decisions in a two-alternative forced choice
task; in neither case is the overall error rate significantly changed.
% Haven't got much more to say here.


\begin{table}[ht]
\caption{Similar to Table~\ref{table:clustering}: Mean latencies (with
  95\% confidence interval) in the three conditions for the entire group, and the three
  identified clusters.}  \centering
\begin{tabular}{c c c c}
\hline
\raisebox{-3ex}{\textbf{Group}} & \multicolumn{3}{c}{\raisebox{-1ex}{\textbf{Latencies for conditions}}} \\ [1ex]
 & Synchronous & No Distractor & Asynchronous \\ [1ex]
\hline
\textbf{\emph{Entire group (N=55)}} & \emph{ 336.0 ms (1.9)} & \emph{295.1 ms (2.1)} &  \emph{345.0 ms (2.9)} \\
\textbf{Faster Synchronous (N=13)}  &        323.5 ms (3.9) &       291.7 ms (4.8) &        371.0 ms (6.3) \\
\textbf{No Difference (N=36)}       &        334.6 ms (2.2) &       292.6 ms (1.9) &        336.5 ms (3.8) \\
\textbf{Faster Asynchronous (N=6)} &        374.7 ms (11.1) &       320.1 ms (5.1) &        340.7 ms (9.1) \\ [1ex]
\hline
\end{tabular}
\label{table:clustering_latencies}
\end{table}
